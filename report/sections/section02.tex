% !TEX root = ../TUCthesis.tex
%*****************************************
\chapter{Getting started}\label{ch:examples}
%*****************************************

As follows, some rules by the authors of \texttt{classicthesis}.

\section{Style Recommendations}

\begin{itemize}
    \item No bold fonts are used. Italics or spaced small caps do the
    job quite well. To use them, enclose your text in \textbackslash emph\{xyz\} or
    \textbackslash spacedlowsmallcaps\{text\} or \textbackslash spacedallcaps\{abc\}.
    \item The tables intentionally do not use vertical or double
    rules. See the documentation for the \texttt{booktabs} package for
    a nice discussion of this topic.\footnote{To be found online at 
    \url{http://mirror.ctan.org/macros/latex/contrib/booktabs/}.}
    A sample table is shown in \autoref{tab:booktabstest} on page\ \pageref{tab:booktabstest}.
\end{itemize}

\begin{table}[t] % t = top, also oben auf der Seite platzieren

\caption{Tolle Tabelle. Sie hat keine Unter-, sondern eine Überschrift!} 
% dies muss immer vor \begin{tabular} stehen, damit es auch eine Überschrift bleibt!

\myfloatalign % zentrieren
\begin{tabular}{cccp{3cm}} % keine vertikalen Linien (d.h. kein | verwenden)
\toprule
\tableheadline{Jahr} & \tableheadline{Monat} & \tableheadline{Tag} & \tableheadline{Ereignis} \\
\midrule
2016 & Juni & 1 & Erstellung dieser Formatvorlage \\
\bottomrule
\end{tabular}
\label{tab:booktabstest} % auf dieses Laben mit \autoref{} verweisen, muss nach dem \end{tabular} stehen
\end{table}

Graphics files can be easily included like the one shown in \autoref{fig:tucseal}.

\begin{figure}[h]
\myfloatalign
\includegraphics[width=0.3\textwidth]{gfx/TUC-seal.pdf}
\caption{Seal of TU Clausthal}
\label{fig:tucseal}
\end{figure}


\section{Organization}
A very important factor for successful thesis writing is the
organization of the material. This version modified for use at TU Clausthal  
suggests a structure as follows:
\begin{itemize}
    \item\texttt{sections/} is where all the ``real'' content goes in
    separate files. Naming is free; you can either enumerate them, such as \texttt{section01.tex}, or
    give your sections descriptive names like \texttt{introdution.tex}. In order to include
    a new section in the document, it must be added in the main matter part of \texttt{TUCthesis.tex}
    by including it (see existing include commands for other sections for reference).

    \item\texttt{FrontBackMatter/} is where all the stuff goes that
    surrounds the ``real'' content, such as the acknowledgments,
    cover page layout, etc.
    
    \item\texttt{gfx/} is where you put all the graphics you use in
    the thesis. Maybe they should be organized into subfolders
    depending on the chapter they are used in, if you have a lot of
    graphics.
    
    \item\texttt{Bibliography.bib}: the Bib\TeX\ database to organize
    all the references you might want to cite.
    
    \item\texttt{classicthesis.sty}: the style definition to get this
    awesome look and feel. 
    
    \item\texttt{TUCthesis.tex}: the main file of your thesis
    where all gets bundled together.
    
    \item\texttt{thesis-config.tex}: a central place to load all 
    nifty packages that are used. You also adjust the title of your thesis, 
    your name, and all  similar information here. 
    
\end{itemize}


\section{Code listings}

\begin{lstlisting}[float=h,language=Java,frame=tb,caption={Beispielcode},label=lst:example]
int j=1;
for (int i=1; i<24; i++) {
  j = j * i;
}
\end{lstlisting}

The code is shown in \autoref{lst:example}
