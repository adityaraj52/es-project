% !TEX root = ../TUCthesis.tex
%************************************************
\chapter{Introduction}\label{ch:introduction}
%************************************************

This bundle for \LaTeX\ has two goals:
\begin{enumerate}
    \item Provide students with an easy-to-use template for their
    Bachelor's or Master's (or even Ph.D.) thesis. 
    \item Provide a classic, high-quality typographic style that is
    inspired by ``\emph{The Elements of
    Typographic Style}'' \cite{bringhurst:2002}.
\end{enumerate}

\bigskip

This package supports acronyms (like \ac{UML}) which are defined in \texttt{Contents.tex}.
On the second use of the acronym (i.e., \ac{UML} again), it will be replaced by the short form.
If you deliberately need it spelled out, use \acl{UML} or \acf{UML}, respectively.

\bigskip

Starting from the second paragraph in a section/chapter, paragraphs are always indented.
You can avoid this by prefixed them with the \texttt{\textbackslash noindent} keyword.
If you want all paragraphs to be indented, configure this in \texttt{thesis-config.tex}.

\bigskip

\noindent If you like the style then its author would appreciate a postcard:
\begin{center}
 André Miede \\
 Detmolder Straße 32 \\
 31737 Rinteln \\
 Germany
\end{center}

You can itemize lists like this:
\begin{itemize}
\item \emph{For example} and \emph{that is} are prefixed and suffixed by a comma, i.e., like this. 
\item Of course this does not apply when starting a sentence. E.g., this is a valid sentence, too.
\end{itemize}

\section{Struktur der Arbeit}
Diese Arbeit ist wie folgt strukturiert.
In \autoref{ch:introduction} gibt es eine Einleitung.
In \autoref{ch:examples} eine Anleitung zu \emph{Getting started}.
Folgend in \autoref{ch:mathtest} ein paar mathematische Beispiele.
Eine Zusammenfassung mit Ausblick findet sich in \autoref{ch:conclusions}.


